\PassOptionsToPackage{unicode=true}{hyperref} % options for packages loaded elsewhere
\PassOptionsToPackage{hyphens}{url}
%
\documentclass[]{article}
\usepackage{lmodern}
\usepackage{amssymb,amsmath}
\usepackage{ifxetex,ifluatex}
\usepackage{fixltx2e} % provides \textsubscript
\ifnum 0\ifxetex 1\fi\ifluatex 1\fi=0 % if pdftex
  \usepackage[T1]{fontenc}
  \usepackage[utf8]{inputenc}
  \usepackage{textcomp} % provides euro and other symbols
\else % if luatex or xelatex
  \usepackage{unicode-math}
  \defaultfontfeatures{Ligatures=TeX,Scale=MatchLowercase}
\fi
% use upquote if available, for straight quotes in verbatim environments
\IfFileExists{upquote.sty}{\usepackage{upquote}}{}
% use microtype if available
\IfFileExists{microtype.sty}{%
\usepackage[]{microtype}
\UseMicrotypeSet[protrusion]{basicmath} % disable protrusion for tt fonts
}{}
\IfFileExists{parskip.sty}{%
\usepackage{parskip}
}{% else
\setlength{\parindent}{0pt}
\setlength{\parskip}{6pt plus 2pt minus 1pt}
}
\usepackage{hyperref}
\hypersetup{
            pdfborder={0 0 0},
            breaklinks=true}
\urlstyle{same}  % don't use monospace font for urls
\usepackage[margin=1in]{geometry}
\usepackage{color}
\usepackage{fancyvrb}
\newcommand{\VerbBar}{|}
\newcommand{\VERB}{\Verb[commandchars=\\\{\}]}
\DefineVerbatimEnvironment{Highlighting}{Verbatim}{commandchars=\\\{\}}
% Add ',fontsize=\small' for more characters per line
\usepackage{framed}
\definecolor{shadecolor}{RGB}{248,248,248}
\newenvironment{Shaded}{\begin{snugshade}}{\end{snugshade}}
\newcommand{\AlertTok}[1]{\textcolor[rgb]{0.94,0.16,0.16}{#1}}
\newcommand{\AnnotationTok}[1]{\textcolor[rgb]{0.56,0.35,0.01}{\textbf{\textit{#1}}}}
\newcommand{\AttributeTok}[1]{\textcolor[rgb]{0.77,0.63,0.00}{#1}}
\newcommand{\BaseNTok}[1]{\textcolor[rgb]{0.00,0.00,0.81}{#1}}
\newcommand{\BuiltInTok}[1]{#1}
\newcommand{\CharTok}[1]{\textcolor[rgb]{0.31,0.60,0.02}{#1}}
\newcommand{\CommentTok}[1]{\textcolor[rgb]{0.56,0.35,0.01}{\textit{#1}}}
\newcommand{\CommentVarTok}[1]{\textcolor[rgb]{0.56,0.35,0.01}{\textbf{\textit{#1}}}}
\newcommand{\ConstantTok}[1]{\textcolor[rgb]{0.00,0.00,0.00}{#1}}
\newcommand{\ControlFlowTok}[1]{\textcolor[rgb]{0.13,0.29,0.53}{\textbf{#1}}}
\newcommand{\DataTypeTok}[1]{\textcolor[rgb]{0.13,0.29,0.53}{#1}}
\newcommand{\DecValTok}[1]{\textcolor[rgb]{0.00,0.00,0.81}{#1}}
\newcommand{\DocumentationTok}[1]{\textcolor[rgb]{0.56,0.35,0.01}{\textbf{\textit{#1}}}}
\newcommand{\ErrorTok}[1]{\textcolor[rgb]{0.64,0.00,0.00}{\textbf{#1}}}
\newcommand{\ExtensionTok}[1]{#1}
\newcommand{\FloatTok}[1]{\textcolor[rgb]{0.00,0.00,0.81}{#1}}
\newcommand{\FunctionTok}[1]{\textcolor[rgb]{0.00,0.00,0.00}{#1}}
\newcommand{\ImportTok}[1]{#1}
\newcommand{\InformationTok}[1]{\textcolor[rgb]{0.56,0.35,0.01}{\textbf{\textit{#1}}}}
\newcommand{\KeywordTok}[1]{\textcolor[rgb]{0.13,0.29,0.53}{\textbf{#1}}}
\newcommand{\NormalTok}[1]{#1}
\newcommand{\OperatorTok}[1]{\textcolor[rgb]{0.81,0.36,0.00}{\textbf{#1}}}
\newcommand{\OtherTok}[1]{\textcolor[rgb]{0.56,0.35,0.01}{#1}}
\newcommand{\PreprocessorTok}[1]{\textcolor[rgb]{0.56,0.35,0.01}{\textit{#1}}}
\newcommand{\RegionMarkerTok}[1]{#1}
\newcommand{\SpecialCharTok}[1]{\textcolor[rgb]{0.00,0.00,0.00}{#1}}
\newcommand{\SpecialStringTok}[1]{\textcolor[rgb]{0.31,0.60,0.02}{#1}}
\newcommand{\StringTok}[1]{\textcolor[rgb]{0.31,0.60,0.02}{#1}}
\newcommand{\VariableTok}[1]{\textcolor[rgb]{0.00,0.00,0.00}{#1}}
\newcommand{\VerbatimStringTok}[1]{\textcolor[rgb]{0.31,0.60,0.02}{#1}}
\newcommand{\WarningTok}[1]{\textcolor[rgb]{0.56,0.35,0.01}{\textbf{\textit{#1}}}}
\usepackage{graphicx,grffile}
\makeatletter
\def\maxwidth{\ifdim\Gin@nat@width>\linewidth\linewidth\else\Gin@nat@width\fi}
\def\maxheight{\ifdim\Gin@nat@height>\textheight\textheight\else\Gin@nat@height\fi}
\makeatother
% Scale images if necessary, so that they will not overflow the page
% margins by default, and it is still possible to overwrite the defaults
% using explicit options in \includegraphics[width, height, ...]{}
\setkeys{Gin}{width=\maxwidth,height=\maxheight,keepaspectratio}
\setlength{\emergencystretch}{3em}  % prevent overfull lines
\providecommand{\tightlist}{%
  \setlength{\itemsep}{0pt}\setlength{\parskip}{0pt}}
\setcounter{secnumdepth}{0}
% Redefines (sub)paragraphs to behave more like sections
\ifx\paragraph\undefined\else
\let\oldparagraph\paragraph
\renewcommand{\paragraph}[1]{\oldparagraph{#1}\mbox{}}
\fi
\ifx\subparagraph\undefined\else
\let\oldsubparagraph\subparagraph
\renewcommand{\subparagraph}[1]{\oldsubparagraph{#1}\mbox{}}
\fi

% set default figure placement to htbp
\makeatletter
\def\fps@figure{htbp}
\makeatother


\author{}
\date{\vspace{-2.5em}}

\begin{document}

\hypertarget{this-will-use-the-tidyverse-suite-of-packages}{%
\subsubsection{This will use the ``tidyverse'' suite of
packages}\label{this-will-use-the-tidyverse-suite-of-packages}}

\begin{Shaded}
\begin{Highlighting}[]
\KeywordTok{library}\NormalTok{(tidyverse)}
\end{Highlighting}
\end{Shaded}

\begin{verbatim}
## ── Attaching packages ─────────────────────────────────────── tidyverse 1.3.0 ──
\end{verbatim}

\begin{verbatim}
## ✓ ggplot2 3.2.1     ✓ purrr   0.3.3
## ✓ tibble  2.1.3     ✓ dplyr   0.8.3
## ✓ tidyr   1.0.0     ✓ stringr 1.4.0
## ✓ readr   1.3.1     ✓ forcats 0.4.0
\end{verbatim}

\begin{verbatim}
## ── Conflicts ────────────────────────────────────────── tidyverse_conflicts() ──
## x dplyr::filter() masks stats::filter()
## x dplyr::lag()    masks stats::lag()
\end{verbatim}

\hypertarget{exercise-1}{%
\subsection{Exercise 1}\label{exercise-1}}

Consider the following:

\begin{Shaded}
\begin{Highlighting}[]
\NormalTok{vector =}\StringTok{ "Good morning! "}
\end{Highlighting}
\end{Shaded}

\textbf{How many characters are in ``vector'' ?}

\hypertarget{exercise-2}{%
\subsection{Exercise 2}\label{exercise-2}}

If:

\begin{Shaded}
\begin{Highlighting}[]
\NormalTok{x <-}\StringTok{ }\KeywordTok{c}\NormalTok{(}\StringTok{"Open"}\NormalTok{, }\StringTok{"Sesame "}\NormalTok{)}
\NormalTok{y <-}\StringTok{ }\KeywordTok{c}\NormalTok{(}\StringTok{"You"}\NormalTok{,}\StringTok{"Suck."}\NormalTok{)}
\KeywordTok{nchar}\NormalTok{(x)}
\end{Highlighting}
\end{Shaded}

\begin{verbatim}
## [1] 4 7
\end{verbatim}

\textbf{Then, what is the value of:}

\begin{Shaded}
\begin{Highlighting}[]
\KeywordTok{nchar}\NormalTok{(}\KeywordTok{c}\NormalTok{(x,y))}
\end{Highlighting}
\end{Shaded}

\hypertarget{exercise-3}{%
\subsection{Exercise 3}\label{exercise-3}}

If:

\begin{Shaded}
\begin{Highlighting}[]
\NormalTok{m <-}\StringTok{ "The capital of the United States is Washington, D.C."}
\KeywordTok{unlist}\NormalTok{(}\KeywordTok{str_split}\NormalTok{(m,}\StringTok{" "}\NormalTok{))}
\end{Highlighting}
\end{Shaded}

\begin{verbatim}
## [1] "The"         "capital"     "of"          "the"         "United"     
## [6] "States"      "is"          "Washington," "D.C."
\end{verbatim}

\ldots{}And:

\begin{Shaded}
\begin{Highlighting}[]
\KeywordTok{str_trunc}\NormalTok{(m,}\DecValTok{11}\NormalTok{, }\DataTypeTok{ellipsis =} \StringTok{""}\NormalTok{)}
\end{Highlighting}
\end{Shaded}

\begin{verbatim}
## [1] "The capital"
\end{verbatim}

\ldots{}And:

\begin{Shaded}
\begin{Highlighting}[]
\KeywordTok{str_sub}\NormalTok{(m,}\DataTypeTok{start =} \DecValTok{13}\NormalTok{,}\DataTypeTok{end =} \DecValTok{25}\NormalTok{)}
\end{Highlighting}
\end{Shaded}

\begin{verbatim}
## [1] "of the United"
\end{verbatim}

\textbf{Come up with a way to extract ``Washington'' from m.}

\hypertarget{exercise-4}{%
\subsection{Exercise 4}\label{exercise-4}}

If:

\begin{Shaded}
\begin{Highlighting}[]
\KeywordTok{paste}\NormalTok{(m,}\StringTok{", you idiot!"}\NormalTok{, }\DataTypeTok{sep =} \StringTok{""}\NormalTok{)}
\end{Highlighting}
\end{Shaded}

\begin{verbatim}
## [1] "The capital of the United States is Washington, D.C., you idiot!"
\end{verbatim}

\textbf{Then come up with a way to use the vector ``m'' to paste
together ``United States, you idiot!''}

\hypertarget{exercise-5}{%
\subsection{Exercise 5}\label{exercise-5}}

If:

\begin{Shaded}
\begin{Highlighting}[]
\NormalTok{q =}\StringTok{ "What is the capital of the United States?"}
\KeywordTok{c}\NormalTok{(q, }\KeywordTok{paste0}\NormalTok{(m,}\StringTok{", you idiot!"}\NormalTok{))}
\end{Highlighting}
\end{Shaded}

\begin{verbatim}
## [1] "What is the capital of the United States?"                       
## [2] "The capital of the United States is Washington, D.C., you idiot!"
\end{verbatim}

\textbf{Then, what will be the value of ``d'' for:}

\begin{Shaded}
\begin{Highlighting}[]
\NormalTok{d =}\StringTok{ }\KeywordTok{str_split}\NormalTok{(}\KeywordTok{c}\NormalTok{(q, }\KeywordTok{paste0}\NormalTok{(m,}\StringTok{", you idiot!"}\NormalTok{)), }\DataTypeTok{pattern =} \StringTok{" "}\NormalTok{)}
\end{Highlighting}
\end{Shaded}

\hypertarget{exercise-6}{%
\subsection{Exercise 6}\label{exercise-6}}

If:

\begin{Shaded}
\begin{Highlighting}[]
\KeywordTok{c}\NormalTok{(}\KeywordTok{unlist}\NormalTok{(}\KeywordTok{map}\NormalTok{(d,}\DecValTok{1}\NormalTok{)),}\StringTok{"Heck!?"}\NormalTok{)}
\end{Highlighting}
\end{Shaded}

\begin{verbatim}
## [1] "What"   "The"    "Heck!?"
\end{verbatim}

And:

\begin{Shaded}
\begin{Highlighting}[]
\KeywordTok{unlist}\NormalTok{(}\KeywordTok{map}\NormalTok{(d,}\DecValTok{2}\NormalTok{))}
\end{Highlighting}
\end{Shaded}

\begin{verbatim}
## [1] "is"      "capital"
\end{verbatim}

\textbf{Then what does map(d,1) do?} \textbf{\ldots{}And why did I wrap
it in unlist()}

\hypertarget{exercise-7}{%
\subsection{Exercise 7}\label{exercise-7}}

If:

\begin{Shaded}
\begin{Highlighting}[]
\NormalTok{t =}\StringTok{ }\KeywordTok{c}\NormalTok{(}\StringTok{"a"}\NormalTok{,}\StringTok{"ab"}\NormalTok{,}\StringTok{"c"}\NormalTok{,}\StringTok{"d"}\NormalTok{,}\StringTok{"e"}\NormalTok{,}\StringTok{"fa"}\NormalTok{)}
\KeywordTok{grep}\NormalTok{(}\StringTok{"a"}\NormalTok{,t)}
\end{Highlighting}
\end{Shaded}

\begin{verbatim}
## [1] 1 2 6
\end{verbatim}

\begin{Shaded}
\begin{Highlighting}[]
\KeywordTok{grepl}\NormalTok{(}\StringTok{"a"}\NormalTok{,t)}
\end{Highlighting}
\end{Shaded}

\begin{verbatim}
## [1]  TRUE  TRUE FALSE FALSE FALSE  TRUE
\end{verbatim}

\ldots{}And:

\begin{Shaded}
\begin{Highlighting}[]
\NormalTok{f =}\StringTok{ }\KeywordTok{c}\NormalTok{(}\StringTok{"b"}\NormalTok{,}\StringTok{"ca"}\NormalTok{,}\StringTok{"at"}\NormalTok{,}\StringTok{"c"}\NormalTok{,}\StringTok{"e"}\NormalTok{,}\StringTok{"aa"}\NormalTok{)}
\NormalTok{v =}\StringTok{ }\KeywordTok{list}\NormalTok{(f,t)}
\NormalTok{v}
\end{Highlighting}
\end{Shaded}

\begin{verbatim}
## [[1]]
## [1] "b"  "ca" "at" "c"  "e"  "aa"
## 
## [[2]]
## [1] "a"  "ab" "c"  "d"  "e"  "fa"
\end{verbatim}

\begin{Shaded}
\begin{Highlighting}[]
\KeywordTok{grep}\NormalTok{(}\StringTok{"a"}\NormalTok{,v)}
\end{Highlighting}
\end{Shaded}

\begin{verbatim}
## [1] 1 2
\end{verbatim}

\begin{Shaded}
\begin{Highlighting}[]
\KeywordTok{grepl}\NormalTok{(}\StringTok{"a"}\NormalTok{,v)}
\end{Highlighting}
\end{Shaded}

\begin{verbatim}
## [1] TRUE TRUE
\end{verbatim}

\textbf{Then what will be the values of the following two expressions?}

\begin{Shaded}
\begin{Highlighting}[]
\KeywordTok{grep}\NormalTok{(}\StringTok{"What"}\NormalTok{,d)}
\KeywordTok{grepl}\NormalTok{(}\StringTok{"What"}\NormalTok{,d)}
\end{Highlighting}
\end{Shaded}

\end{document}
